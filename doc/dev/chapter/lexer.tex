% !TeX root = ../dev.tex

该部分\footnote{该部分将会被用于编译原理课程的第5次作业第2题。}旨在建立对于 \{\LaTeXe{} 命令\} -- \{\TeX{} 命令\} 子集的词法分析器(lexer)。这些命令用于 \textsf{AutoBeamer} 识别文档分块。

\section{词法单元}

\textsf{AutoBeamer} 主要识别两种基本的词法单元,正则描述如表 \ref{tab:basic-unit} 所示:
\begin{itemize}
    \item 命令(比如:\verb"\title{Slide Title}",命令参数是可选的)
    \item 英文词(比如:\verb"beamer")
    \item 其他语言的单字(比如:幻,灯,片)
\end{itemize}

\begin{table}[h]
    \centering
    \caption{\textsf{AutoBeamer} 所识别的基本词法单元}
    \label{tab:basic-unit}
    \begin{tabular}{>{\bfseries}c|cl}
        \toprule
        词法单元 & 非正式描述 & 正则描述\\
        \midrule
        command & 命令  & \verb"\\\w+(\[.*\]|\{.*\})*" \\
        word   & 词  & \verb"\w+" \\
        character & 字 & \verb"[^(\\|\w|\s)]" \\
        \bottomrule
    \end{tabular}
\end{table}

此外,对于命令语法单元,\textsf{AutoBeamer} 需要进行扩展识别,其相关正则描述如表 \ref{tab:extend-unit} 所示:
\begin{itemize}
    \item 环境开始(比如:\verb"\begin{document}")
    \item 环境结束(比如:\verb"\end{document}")
    \item 文档类(比如: \verb"\documentclass{article}")
\end{itemize}

\begin{table}[h]
    \centering
    \caption{\textsf{AutoBeamer} 所识别的命令扩展词法单元}
    \label{tab:extend-unit}
    \begin{tabular}{>{\bfseries}c|cl}
        \toprule
        词法单元 & 非正式描述 & 正则描述\\
        \midrule
        begin   & 环境开始  & \verb"\\begin{\w+}" \\
        end     & 环境结束  & \verb"\\end{\w+}" \\
        class   & 文档类    & \verb"\\documentclass(\[\w*\])?{\w+}" \\
        \bottomrule
    \end{tabular}
\end{table}

\section{构建 DFA}

确定的有穷自动机(Determinstic Finite Automata, DFA) 是构建词法分析器的重要依据。

基本的词法单元需要先识别命令,再识别词/字,其 DFA 如图 \ref{fig:basic-dfa} 所示。注意,输入的文件应该是已经可以通过 \LaTeX{} 编译的普通文档(不会出现 Error 异常,能输出 PDF 文档),所以 \verb"\" 后面应该就是字母,否则会报错。

\begin{figure}[h]
    \centering
    \begin{tikzpicture}[shorten >=1pt]
        \node[initial,state] (v1) at (-2,1) {0};
        \node[state] (v2) at (0.5,3) {1};
        \draw  (v1) edge[->] node[above] {$\backslash\backslash$} (v2);
        \node[accepting,state] (v4) at (3,3) {2};
        \draw  (v2) edge[->] node[above] {$\backslash$w} (v4);
        \draw (v4) edge[loop above] node {$\backslash$w} (v4);
        \node[state] (v5) at (5,4.5) {3};
        \draw  (v4) edge[->,bend left] node[above] {$\backslash$[} (v5);
        \node[state] (v6) at (7,5.5) {4};
        \draw  (v5) edge[->] node[above] {[$^\land\backslash$]]} (v6);
        
        \draw  (v6) edge[loop below] node {[$^\land\backslash$]]} (v6);
        \draw  (v5) edge[->,bend left] node[above] {$\backslash$]} (v4);
        \draw  (v6) edge[->,bend left] node[above] {$\backslash$]} (v4);
        \node[state] (v7) at (5,1.5) {5};
        \draw  (v4) edge[->,bend left] node[below] {$\backslash$\{} (v7);
        \draw  (v7) edge[->,bend left] node[below] {$\backslash$\}} (v4);
        \node[state] (v8) at (7,0.5) {6};
        \draw  (v7) edge[->] node[below] {[$^\land\backslash$\}]} (v8);
        \draw  (v8) edge[->,bend right] node[below] {$\backslash$\}} (v4);
        \draw  (v8) edge[loop above] node[above] {[$^\land\backslash$\}]}  (v8);
        
        \node[accepting,state] (v3) at (0.5,1) {7};
        \draw  (v1) edge[->] node[above] {$\backslash$w} (v3);
        \draw  (v3) edge[loop right] node {$\backslash$w} (v3);
        \node[accepting,state] (v9) at (0.5,-1) {8};
        \draw  (v1) edge[->] node[left] {[$\rm ^\land(\backslash\backslash|\backslash w|\backslash s)$]} (v9);
        
        \node [below left of=v4,font=\bfseries] {command};
        \node [below right of=v3,font=\bfseries] {word};
        \node [below right of=v9,font=\bfseries] {character};
    \end{tikzpicture}
    \caption{基本词法单元的 DFA}
    \label{fig:basic-dfa}
\end{figure}


对于命令扩展词法单元而言,需要有其已经是命令的先决条件,其 DFA 如图 \ref{fig:extend-dfa} 所示。仍然是在文档是编译正确的前提下进行的,如果没有匹配到扩展命令,将仍然作为 \textbf{command} 词法单元。

\begin{figure}[h]
    \centering
    \begin{tikzpicture}[shorten >=1pt]
        \node[initial,state] (v1) at (0,0) {0};
        \node[state] (v2) at (2,0) {1};
        \draw[->]  (v1) edge node [above] {$\backslash\backslash$} (v2);
        \node[state] (v3) at (4,2) {2};
        
        \node[state] (v4) at (6,2) {3};
        \node[state] (v5) at (8,2) {4};
        \node[accepting,state] (v6) at (10,2) {5};
        \draw  (v2) edge[->] node[left,font=\ttfamily] {begin} (v3);
        \draw  (v3) edge[->] node[above] {$\backslash$\{} (v4);
        \draw  (v4) edge[->] node[above] {$\backslash$w} (v5);
        \draw  (v5) edge[loop above] node {$\backslash$w} (v5);
        \draw  (v5) edge[->] node[above] {$\backslash$\}} (v6);
        
        \node[state] (v7) at (4,0) {6};
        \node[state] (v8) at (6,0) {7};
        \node[state] (v9) at (8,0) {8};
        \node[accepting,state] (v10) at (10,0) {9};
        \draw  (v2) edge [->] node[above,font=\ttfamily] {end} (v7);
        \draw  (v7) edge [->] node[above] {$\backslash$\{} (v8);
        \draw  (v8) edge [->] node[above] {$\backslash$w}  (v9);
        \draw  (v9) edge [->] node[above] {$\backslash$\}}  (v10);
        \draw  (v9) edge[loop above]  node {$\backslash$w} (v9);
        \node[state] (v11) at (4,-2) {10};
        \node[state] (v12) at (6,-2) {11};
        \node[state] (v15) at (8,-2) {12};
        \node[accepting,state] (v16) at (10,-2) {13};
        \node[state] (v13) at (4,-3.5) {14};
        \node[state] (v14) at (6,-3.5) {15};
        \draw  (v2) edge[->] node[left,font=\ttfamily] {documentclass}(v11);
        \draw  (v11) edge[->] node[above] {$\backslash$\{}(v12);
        \draw  (v11) edge[->] node[left] {$\backslash$[}(v13);
        \draw  (v13) edge[->] node[above] {$\backslash$]}(v14);
        \draw  (v14) edge[->] node[right] {$\backslash$\{}(v12);
        \draw  (v12) edge[->] node[above] {$\backslash$w}(v15);
        \draw  (v15) edge[->] node[above] {$\backslash$\}}(v16);
        \draw  (v13) edge[->,loop left] node[left] {$\backslash$w} (v13);
        \draw  (v15) edge[->,loop above] node[above] {$\backslash$w}  (v15);
        \node[accepting,state] (v17) at (0,2) {16};
        \draw  (v2) edge[->] node[font=\itshape,left] {other} (v17);
        
        \node[below right of=v6,font=\bfseries] {begin};
        \node[below right of=v10,font=\bfseries] {end};
        \node[below right of=v16,font=\bfseries] {class};
        \node[above right of=v17,font=\bfseries] {command};
    \end{tikzpicture}
    \caption{命令扩展词法单元的 DFA}
    \label{fig:extend-dfa}
\end{figure}

